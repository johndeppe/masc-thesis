%%%%%%%%%%%%%%%%%%%%%%%%%%%%%%%%%%%%%%%%%%%%%%%%%%%%%%%%%%%%%%%%%%%%%%
% Template for a UBC-compliant dissertation
% At the minimum, you will need to change the information found
% after the "Document meta-data"
%
%!TEX TS-program = pdflatex
%!TEX encoding = UTF-8 Unicode

%% The ubcdiss class provides several options:
%%   gpscopy (aka fogscopy)
%%       set parameters to exactly how GPS specifies
%%         * single-sided
%%         * page-numbering starts from title page
%%         * the lists of figures and tables have each entry prefixed
%%           with 'Figure' or 'Table'
%%       This can be tested by `\ifgpscopy ... \else ... \fi'
%%   10pt, 11pt, 12pt
%%       set default font size
%%   oneside, twoside
%%       whether to format for single-sided or double-sided printing
%%   balanced
%%       when double-sided, ensure page content is centred
%%       rather than slightly offset (the default)
%%   singlespacing, onehalfspacing, doublespacing
%%       set default inter-line text spacing; the ubcdiss class
%%       provides \textspacing to revert to this configured spacing
%%   draft
%%       disable more intensive processing, such as including
%%       graphics, etc.
%%

% For submission to GPS
\documentclass[gpscopy,onehalfspacing,11pt]{ubcdiss}

% For your own copies (looks nicer)
% \documentclass[balanced,twoside,11pt]{ubcdiss}

%%%%%%%%%%%%%%%%%%%%%%%%%%%%%%%%%%%%%%%%%%%%%%%%%%%%%%%%%%%%%%%%%%%%%%
%%%%%%%%%%%%%%%%%%%%%%%%%%%%%%%%%%%%%%%%%%%%%%%%%%%%%%%%%%%%%%%%%%%%%%
%%
%% FONTS:
%% 
%% The defaults below configures Times Roman for the serif font,
%% Helvetica for the sans serif font, and Courier for the
%% typewriter-style font.  Configuring fonts can be time
%% consuming; we recommend skipping to END FONTS!
%% 
%% If you're feeling brave, have lots of time, and wish to use one
%% your platform's native fonts, see the commented out bits below for
%% XeTeX/XeLaTeX.  This is not for the faint at heart. 
%% (And shouldn't you be writing? :-)
%%

%% NFSS font specification (New Font Selection Scheme)
\usepackage{times,mathptmx,courier}
\usepackage[scaled=.92]{helvet}

%% Math or theory people may want to include the handy AMS macros
%\usepackage{amssymb}
%\usepackage{amsmath}
%\usepackage{amsfonts}

%% The pifont package provides access to the elements in the dingbat font.   
%% Use \ding{##} for a particular dingbat (see p7 of psnfss2e.pdf)
%%   Useful:
%%     51,52 different forms of a checkmark
%%     54,55,56 different forms of a cross (saltyre)
%%     172-181 are 1-10 in open circle (serif)
%%     182-191 are 1-10 black circle (serif)
%%     192-201 are 1-10 in open circle (sans serif)
%%     202-211 are 1-10 in black circle (sans serif)
%% \begin{dinglist}{##}\item... or dingautolist (which auto-increments)
%% to create a bullet list with the provided character.
\usepackage{pifont}

%%%%%%%%%%%%%%%%%%%%%%%%%%%%%%%%%%%%%%%%%%%%%%%%%%%%%%%%%%%%%%%%%%%%%%
%% Configure fonts for XeTeX / XeLaTeX using the fontspec package.
%% Be sure to check out the fontspec documentation.
%\usepackage{fontspec,xltxtra,xunicode}	% required
%\defaultfontfeatures{Mapping=tex-text}	% recommended
%% Minion Pro and Myriad Pro are shipped with some versions of
%% Adobe Reader.  Adobe representatives have commented that these
%% fonts can be used outside of Adobe Reader.
%\setromanfont[Numbers=OldStyle]{Minion Pro}
%\setsansfont[Numbers=OldStyle,Scale=MatchLowercase]{Myriad Pro}
%\setmonofont[Scale=MatchLowercase]{Andale Mono}

%% Other alternatives:
%\setromanfont[Mapping=tex-text]{Adobe Caslon}
%\setsansfont[Scale=MatchLowercase]{Gill Sans}
%\setsansfont[Scale=MatchLowercase,Mapping=tex-text]{Futura}
%\setmonofont[Scale=MatchLowercase]{Andale Mono}
%\newfontfamily{\SYM}[Scale=0.9]{Zapf Dingbats}
%% END FONTS
%%%%%%%%%%%%%%%%%%%%%%%%%%%%%%%%%%%%%%%%%%%%%%%%%%%%%%%%%%%%%%%%%%%%%%
%%%%%%%%%%%%%%%%%%%%%%%%%%%%%%%%%%%%%%%%%%%%%%%%%%%%%%%%%%%%%%%%%%%%%%



%%%%%%%%%%%%%%%%%%%%%%%%%%%%%%%%%%%%%%%%%%%%%%%%%%%%%%%%%%%%%%%%%%%%%%
%%%%%%%%%%%%%%%%%%%%%%%%%%%%%%%%%%%%%%%%%%%%%%%%%%%%%%%%%%%%%%%%%%%%%%
%%
%% Recommended packages
%%
\usepackage{checkend}	% better error messages on left-open environments
\usepackage{graphicx}	% for incorporating external images

%% booktabs: provides some special commands for typesetting tables as used
%% in excellent journals.  Ignore the examples in the Lamport book!
\usepackage{booktabs}

%% listings: useful support for including source code listings, with
%% optional special keyword formatting.  The \lstset{} causes
%% the text to be typeset in a smaller sans serif font, with
%% proportional spacing.
\usepackage{listings}
\lstset{basicstyle=\sffamily\scriptsize,showstringspaces=false,fontadjust}

%% The acronym package provides support for defining acronyms, providing
%% their expansion when first used, and building glossaries.  See the
%% example in glossary.tex and the example usage throughout the example
%% document.
%% NOTE: to use \MakeTextLowercase in the \acsfont command below,
%%   we *must* use the `nohyperlinks' option -- it causes errors with
%%   hyperref otherwise.  See Section 5.2 in the ``LaTeX 2e for Class
%%   and Package Writers Guide'' (clsguide.pdf) for details.
\usepackage[printonlyused,nohyperlinks]{acronym}
%% The ubcdiss.cls loads the `textcase' package which provides commands
%% for upper-casing and lower-casing text.  The following causes
%% the acronym package to typeset acronyms in small-caps
%% as recommended by Bringhurst.
\renewcommand{\acsfont}[1]{{\scshape \MakeTextLowercase{#1}}}

%% color: add support for expressing colour models.  Grey can be used
%% to great effect to emphasize other parts of a graphic or text.
%% For an excellent set of examples, see Tufte's "Visual Display of
%% Quantitative Information" or "Envisioning Information".
\usepackage{color}
\definecolor{greytext}{gray}{0.5}

%% comment: provides a new {comment} environment: all text inside the
%% environment is ignored.
%%   \begin{comment} ignored text ... \end{comment}
\usepackage{comment}

%% The natbib package provides more sophisticated citing commands
%% such as \citeauthor{} to provide the author names of a work,
%% \citet{} to produce an author-and-reference citation,
%% \citep{} to produce a parenthetical citation.
%% We use \citeeg{} to provide examples
\usepackage[numbers,sort&compress]{natbib}
\newcommand{\citeeg}[1]{\citep[e.g.,][]{#1}}

%% The titlesec package provides commands to vary how chapter and
%% section titles are typeset.  The following uses more compact
%% spacings above and below the title.  The titleformat that follow
%% ensure chapter/section titles are set in singlespace.
\usepackage[compact]{titlesec}
\titleformat*{\section}{\singlespacing\raggedright\bfseries\Large}
\titleformat*{\subsection}{\singlespacing\raggedright\bfseries\large}
\titleformat*{\subsubsection}{\singlespacing\raggedright\bfseries}
\titleformat*{\paragraph}{\singlespacing\raggedright\itshape}

%% The caption package provides support for varying how table and
%% figure captions are typeset.
\usepackage[format=hang,indention=-1cm,labelfont={bf},margin=1em]{caption}

%% url: for typesetting URLs and smart(er) hyphenation.
%% \url{http://...} 
\usepackage{url}
\urlstyle{sf}	% typeset urls in sans-serif


%%%%%%%%%%%%%%%%%%%%%%%%%%%%%%%%%%%%%%%%%%%%%%%%%%%%%%%%%%%%%%%%%%%%%%
%%%%%%%%%%%%%%%%%%%%%%%%%%%%%%%%%%%%%%%%%%%%%%%%%%%%%%%%%%%%%%%%%%%%%%
%%
%% Possibly useful packages: you may need to explicitly install
%% these from CTAN if they aren't part of your distribution;
%% teTeX seems to ship with a smaller base than MikTeX and MacTeX.
%%
%\usepackage{pdfpages}	% insert pages from other PDF files
%\usepackage{longtable}	% provide tables spanning multiple pages
%\usepackage{chngpage}	% support changing the page widths on demand
%\usepackage{tabularx}	% an enhanced tabular environment

%% enumitem: support pausing and resuming enumerate environments.
%\usepackage{enumitem}

%% rotating: provides two environments, sidewaystable and sidewaysfigure,
%% for typesetting tables and figures in landscape mode.  
%\usepackage{rotating}

%% subfig: provides for including subfigures within a figure,
%% and includes being able to separately reference the subfigures.
%\usepackage{subfig}

%% ragged2e: provides several new new commands \Centering, \RaggedLeft,
%% \RaggedRight and \justifying and new environments Center, FlushLeft,
%% FlushRight and justify, which set ragged text and are easily
%% configurable to allow hyphenation.
%\usepackage{ragged2e}

%% The ulem package provides a \sout{} for striking out text and
%% \xout for crossing out text.  The normalem and normalbf are
%% necessary as the package messes with the emphasis and bold fonts
%% otherwise.
%\usepackage[normalem,normalbf]{ulem}    % for \sout

%%%%%%%%%%%%%%%%%%%%%%%%%%%%%%%%%%%%%%%%%%%%%%%%%%%%%%%%%%%%%%%%%%%%%%
%% HYPERREF:
%% The hyperref package provides for embedding hyperlinks into your
%% document.  By default the table of contents, references, citations,
%% and footnotes are hyperlinked.
%%
%% Hyperref provides a very handy command for doing cross-references:
%% \autoref{}.  This is similar to \ref{} and \pageref{} except that
%% it automagically puts in the *type* of reference.  For example,
%% referencing a figure's label will put the text `Figure 3.4'.
%% And the text will be hyperlinked to the appropriate place in the
%% document.
%%
%% Generally hyperref should appear after most other packages

%% The `pagebackref' causes the references in the bibliography to have
%% back-references to the citing page; `backref' puts the citing section
%% number.  See further below for other examples of using hyperref.
%% 2009/12/09: now use `linktocpage' (Jacek Kisynski): GPS now prefers
%%   that the ToC, LoF, LoT place the hyperlink on the page number,
%%   rather than the entry text.
\ifgpscopy
  % GPS requires that weblinks should be dark blue, which looks a bit
  % odd in printed form.
  % https://www.grad.ubc.ca/current-students/dissertation-thesis-preparation/fonts-print
  \usepackage[bookmarks,bookmarksnumbered,%
     pagebackref,linktocpage,%
     colorlinks=true,%
     linkcolor=black,%
     urlcolor=blue,%
     citecolor=black%
     ]{hyperref}
\else
  %% The following puts hyperlinks in very faint grey boxes (in pdf only).
  \usepackage[bookmarks,bookmarksnumbered,%
    pagebackref,linktocpage,%
    allbordercolors={0.8 0.8 0.8},%
    ]{hyperref}
\fi
%% The following change how the the back-references text is typeset in a
%% bibliography when `backref' or `pagebackref' are used
%%
%% Change \nocitations if you'd like some text shown where there
%% are no citations found (e.g., pulled in with \nocite{xxx})
\newcommand{\nocitations}{\relax}
%%\newcommand{\nocitations}{No citations}
%%
%\renewcommand*{\backref}[1]{}% necessary for backref < 1.33
\renewcommand*{\backrefsep}{,~}%
\renewcommand*{\backreftwosep}{,~}% ', and~'
\renewcommand*{\backreflastsep}{,~}% ' and~'
\renewcommand*{\backrefalt}[4]{%
\textcolor{greytext}{\ifcase #1%
\nocitations%
\or
\(\rightarrow\) page #2%
\else
\(\rightarrow\) pages #2%
\fi}}


%% The following uses most defaults, which causes hyperlinks to be
%% surrounded by colourful boxes; the colours are only visible in
%% PDFs and don't show up when printed:
%\usepackage[bookmarks,bookmarksnumbered]{hyperref}

%% The following disables the colourful boxes around hyperlinks.
%\usepackage[bookmarks,bookmarksnumbered,pdfborder={0 0 0}]{hyperref}

%% The following disables all hyperlinking, but still enabled use of
%% \autoref{}
%\usepackage[draft]{hyperref}

%% The following commands causes chapter and section references to
%% uppercase the part name.
\renewcommand{\chapterautorefname}{Chapter}
\renewcommand{\sectionautorefname}{Section}
\renewcommand{\subsectionautorefname}{Section}
\renewcommand{\subsubsectionautorefname}{Section}

%% If you have long page numbers (e.g., roman numbers in the 
%% preliminary pages for page 28 = xxviii), you might need to
%% uncomment the following and tweak the \@pnumwidth length
%% (default: 1.55em).  See the tocloft documentation at
%% http://www.ctan.org/tex-archive/macros/latex/contrib/tocloft/
% \makeatletter
% \renewcommand{\@pnumwidth}{3em}
% \makeatother

%%%%%%%%%%%%%%%%%%%%%%%%%%%%%%%%%%%%%%%%%%%%%%%%%%%%%%%%%%%%%%%%%%%%%%
%%%%%%%%%%%%%%%%%%%%%%%%%%%%%%%%%%%%%%%%%%%%%%%%%%%%%%%%%%%%%%%%%%%%%%
%%
%% Some special settings that controls how text is typeset
%%
% \raggedbottom		% pages don't have to line up nicely on the last line
% \sloppy		% be a bit more relaxed in inter-word spacing
% \clubpenalty=10000	% try harder to avoid orphans
% \widowpenalty=10000	% try harder to avoid widows
% \tolerance=1000

%% And include some of our own useful macros
\input{macros}

%%%%%%%%%%%%%%%%%%%%%%%%%%%%%%%%%%%%%%%%%%%%%%%%%%%%%%%%%%%%%%%%%%%%%%
%%%%%%%%%%%%%%%%%%%%%%%%%%%%%%%%%%%%%%%%%%%%%%%%%%%%%%%%%%%%%%%%%%%%%%
%%
%% Document meta-data: be sure to also change the \hypersetup information
%%

\title{TLB Coherence}
%\subtitle{If you want a subtitle}

\author{John Henry Deppe}
\previousdegree{A.Sc, Everett Community College, 2011}
\previousdegree{B.ASc, University of British Columbia (Vancouver), 2019}

% What is this dissertation for?
\degreetitle{Master of Applied Science}

\institution{The University of British Columbia}
\campus{Vancouver}

\faculty{The Faculty of Graduate and Postdoctoral Studies}
\department{Electrical and Computer Engineering}
\submissionmonth{April}
\submissionyear{2025}

% details of your examining committee
\examiningcommittee{Guy Lemieux, Professor, Electrical and Computer Engineering, \textsc{UBC}}{Supervisor}
\examiningcommittee{Mary Maker, Professor, Materials Engineering, \textsc{UBC}}%
    {Supervisory Committee Member}
\examiningcommittee{Nebulous Name, Position, Department, Institution}{Supervisory Committee Member}
\examiningcommittee{Magnus Monolith, Position, Other Department, Institution}{Additional Examiner}

% details of your supervisory committee
\supervisorycommittee{Ira Crater, Professor, Materials Engineering, \textsc{UBC}}%
    {Supervisory Committee Member}

%% hyperref package provides support for embedding meta-data in .PDF
%% files
\hypersetup{
  pdftitle={TLB Coherence (DRAFT: \today)},
  pdfauthor={John Henry Deppe},
  pdfkeywords={TLB Coherence}
}

%%%%%%%%%%%%%%%%%%%%%%%%%%%%%%%%%%%%%%%%%%%%%%%%%%%%%%%%%%%%%%%%%%%%%%
%%%%%%%%%%%%%%%%%%%%%%%%%%%%%%%%%%%%%%%%%%%%%%%%%%%%%%%%%%%%%%%%%%%%%%
%% 
%% The document content
%%

%% LaTeX's \includeonly commands causes any uses of \include{} to only
%% include files that are in the list.  This is helpful to produce
%% subsets of your thesis (e.g., for committee members who want to see
%% the dissertation chapter by chapter).  It also saves time by 
%% avoiding reprocessing the entire file.
%\includeonly{intro,conclusions}
%\includeonly{discussion}

\begin{document}

%%%%%%%%%%%%%%%%%%%%%%%%%%%%%%%%%%%%%%%%%%%%%%%%%%
%% From Thesis Components: Tradtional Thesis
%% <http://www.grad.ubc.ca/current-students/dissertation-thesis-preparation/order-components>

% Preliminary Pages (numbered in lower case Roman numerals)
%    1. Title page (mandatory)
\maketitle

%    2. Committee page (mandatory): lists supervisory committee and,
%    if applicable, the examining committee
\makecommitteepage

%    3. Abstract (mandatory - maximum 350 words)
\include{abstract}
\cleardoublepage

%    4. Lay Summary (Effective May 2017, mandatory - maximum 150 words)
\include{laysummary}
\cleardoublepage

%    5. Preface
\include{preface}
\cleardoublepage

%    6. Table of contents (mandatory - list all items in the preliminary pages
%    starting with the abstract, followed by chapter headings and
%    subheadings, bibliographies and appendices)
\tableofcontents
\cleardoublepage	% required by tocloft package

%    7. List of tables (mandatory if thesis has tables)
\listoftables
\cleardoublepage	% required by tocloft package

%    8. List of figures (mandatory if thesis has figures)
\listoffigures
\cleardoublepage	% required by tocloft package

%    9. List of illustrations (mandatory if thesis has illustrations)
%   10. Lists of symbols, abbreviations or other (optional)

%   11. Glossary (optional)
%% The following is a directive for TeXShop to indicate the main file
%%!TEX root = diss.tex

\chapter{Glossary}

This glossary uses the handy \latexpackage{acroynym} package to automatically
maintain the glossary.  It uses the package's \texttt{printonlyused}
option to include only those acronyms explicitly referenced in the
\LaTeX\ source.  To change how the acronyms are rendered, change the
\verb+\acsfont+ definition in \verb+diss.tex+.

% use \acrodef to define an acronym, but no listing
\acrodef{UI}{user interface}
\acrodef{UBC}{University of British Columbia}

% The acronym environment will typeset only those acronyms that were
% *actually used* in the course of the document
\begin{acronym}[ANOVA]
\acro{TLB}[TLB]{Translation lookaside bluffer\acroextra{a cache for virtual memory translations}}
\acro{ANOVA}[ANOVA]{Analysis of Variance\acroextra{, a set of
  statistical techniques to identify sources of variability between groups}}
\acro{API}{application programming interface}
\acro{CTAN}{\acroextra{The }Common \TeX\ Archive Network}
\acro{DOI}{Document Object Identifier\acroextra{ (see
    \url{http://doi.org})}}
\acro{GPS}[GPS]{Graduate and Postdoctoral Studies}
\acro{PDF}{Portable Document Format}
\acro{RCS}[RCS]{Revision control system\acroextra{, a software
    tool for tracking changes to a set of files}}
\acro{TLX}[TLX]{Task Load Index\acroextra{, an instrument for gauging
  the subjective mental workload experienced by a human in performing
  a task}}
\acro{UML}{Unified Modelling Language\acroextra{, a visual language
    for modelling the structure of software artefacts}}
\acro{URL}{Unique Resource Locator\acroextra{, used to describe a
    means for obtaining some resource on the world wide web}}
\acro{W3C}[W3C]{\acroextra{the }World Wide Web Consortium\acroextra{,
    the standards body for web technologies}}
\acro{XML}{Extensible Markup Language}
\end{acronym}

% You can also use \newacro{}{} to only define acronyms
% but without explictly creating a glossary
% 
% \newacro{ANOVA}[ANOVA]{Analysis of Variance\acroextra{, a set of
%   statistical techniques to identify sources of variability between groups.}}
% \newacro{API}[API]{application programming interface}
% \newacro{GOMS}[GOMS]{Goals, Operators, Methods, and Selection\acroextra{,
%   a framework for usability analysis.}}
% \newacro{TLX}[TLX]{Task Load Index\acroextra{, an instrument for gauging
%   the subjective mental workload experienced by a human in performing
%   a task.}}
% \newacro{UI}[UI]{user interface}
% \newacro{UML}[UML]{Unified Modelling Language}
% \newacro{W3C}[W3C]{World Wide Web Consortium}
% \newacro{XML}[XML]{Extensible Markup Language}
	% always input, since other macros may rely on it

\textspacing		% begin one-half or double spacing

%   12. Acknowledgements (optional)
%% The following is a directive for TeXShop to indicate the main file
%%!TEX root = diss.tex

\chapter{Acknowledgments}

Thank those people who helped you. 

Don't forget your parents or loved ones.

You may wish to acknowledge your funding sources.

Professors Sasha Fedorova, Margo Seltzer, and Reto Achermann taught me enough about operating systems that this project ended up being a part of one instead of a hardware augmentation. Joel Nider was a valuable peer resource as I started the 

Professor Mieszko Lis led me to computer architecture and took a chance on me as a graduate student.

Andes Technology funded me after I returned from leave

Wife, children, veterans administration

Sathish, Farshid, Steve Wolfman all took chances on me and I appreciate their support.

%   13. Dedication (optional)

% Body of Thesis (not all sections may apply)
\mainmatter

\acresetall	% reset all acronyms used so far

%    1. Introduction
%% The following is a directive for TeXShop to indicate the main file
%%!TEX root = diss.tex

\chapter{Introduction}
\label{ch:Introduction}

\begin{epigraph}
	\emph{Perfection is achieved, not when there is nothing more to add, but when there is nothing left to take away.}\\
	---~Antoine de Saint-Exupéry, Airman's Odyssey
\end{epigraph}

Maintaining TLB coherence is required for correctness but presents system overheads that impede multi-core scaling. One approach to reduce TLB coherence  overhead is to reduce TLB shootdown costs by sending shootdown messages to only those TLBs that contained the now-invalid page mapping. We investigate the costs of sharer lists for the purpose of shootdown filtering and propose methods for their use.

\endinput


%    2. Main body
% Generally recommended to put each chapter into a separate file
%% The following is a directive for TeXShop to indicate the main file
%%!TEX root = diss.tex

\chapter{Background}
\label{ch:Background}

\begin{epigraph}
	\emph{Universal law is for lackeys; context is for kings.}\\
	---~Captain Gabriel Lorca, Star Trek: Discovery
\end{epigraph}

TLB coherence has been a thorn in the side of OS writers since the TLB was invented. Patricia Teller outlines the problem nicely, which is that computer manufacturers of the 80s found TLB coherence to be too expensive, and delegated the problem to software. \cite{teller_cost_1990}. \textit{TLB Shootdown}~\cite{black_translation_1989} has become the industry-standard procedure to maintain TLB coherence in popular multi-core architectures including x86, ARM, and RISC-V.

Page tables are typically multi-level radix trees, but other organizations are also used~\cite{jacob_look_1998}. Linux's design uses radix page tables as a unifying approach to managing virtual memory across architectures~\cite{torvalds_linux_1997}.

POWER, ARM, and some RISC-V vendor extensions have inter-core TLB invalidations that don't require IPIs. IPIs create undesirable overhead but even ARM's hardware shootdown can suffer by broadcasting to uninvolved cores~\cite{takao_indoh_patch_2019}.

Intel has recently implemented a hardware TLB invalidation~\cite{intel_corporation_remote_2021} but nobody is using it anywhere as far as I can tell?

With virtualization and hypervisors, maintaining TLB coherence has become complex.

ASIDs, page permissions, weak memory models and races.

TLB shootdowns lead to latency spikes and jitter, \cite{rigtorp_latency_2020, gallenmuller_ducked_2021, gallenmuller_how_2022}. Such latencies present limiting design concerns for systems that oversubscribe memory, including databases~\cite{crotty_are_2022} and fine-grained protection \cite{porter_decker_2023}.

The complex interface leads to operating system bugs~\cite{wong_tlb_2015}, including critical vulnerabilities~\cite{horn_project_2019}.

RISC-V is still emerging, the Svvptc extension lets many sfence.vma instructions be elided~\cite{ghiti_patch_2024}. There's also Svadu and Svinval. Confusing! I'll have to write more about how RISC-V transistency works.

\endinput
%\include{model}
%% The following is a directive for TeXShop to indicate the main file
%%!TEX root = diss.tex

\chapter{Implementing Smokewagon}
\label{ch:Implementation}

\begin{epigraph}
	\emph{Go ahead. Skin it. Skin that smoke wagon and see what happens.} \\
	---~Wyatt Earp, Tombstone
\end{epigraph}

Inspired by ABIS \cite{amit_optimizing_2017}, we base our efforts on the Linux kernel.

\endinput
%%% The following is a directive for TeXShop to indicate the main file
%%!TEX root = diss.tex

\chapter{Performance and Discussion}
\label{ch:discussion}

\begin{epigraph}
	\emph{Found it. There's your problem.}\\
	---~Harry Tuttle, Brazil
\end{epigraph}

\section{Benchmarks}

page-unmapping microbenchmark

apache

oversubscribed database


\section{Related Work}
\textbf{Page Access Tracking:} ABIS~\cite{amit_optimizing_2017} tracks pages accessed for the purpose of reducing shootdown IPI overhead.

\textbf{Software Shootdown Batching:} LATR~\cite{kumar_latr_2018} and EcoTLB~\cite{maass_ecotlb_2020} delay and batch shootdowns that can be delayed, such as memory deallocation.

\textbf{Software Hugepage Creation:} TridentPV~\cite{ram_trident_2021} has a nice hypercall trick to create guest hugepages quickly that improved TLB shootdowns will make faster.

\textbf{RISC-V Caches and Coherence:} Skip It~\cite{anand_skip_2024} avoids unnecessary writebacks with special instructions implemented in BOOM.

Svinval extension allows TLB invalidation without a complete fence.

Svvptc allows eliding some sfence.vmas \url{https://lore.kernel.org/lkml/Zbuy1E7mz9Oui1Dl@andrea/T/#m34eae90356460ae5fc28b0e9ff27b14fe24413e0}

A clue for how to manage speculative page faults while mucking about with page tables? \url{https://lore.kernel.org/linux-riscv/20230801090927.2018653-1-dylan@andestech.com/}

\textbf{Hardware TLB Coherence:}
UNITD~\cite{romanescu_unified_2010}
HATRIC~\cite{yan_hardware_2017}
DiDi~\cite{villavieja_didi_2011}

ATTC~\cite{gugale_attc_2020} manages shootdowns from both guests and hosts by using writes to an in-DRAM TLB to trigger coherence actions. This requires modifying the guests to use such a para-virtualized scheme. PIN trace-based simulation.

\textbf{etc:} CC-NIC provides reduces NIC latency with cache-coherence~\cite{schuh_cc-nic_2024}. The producer-consumer workload of NICs presents another case where write-updates might be useful.

\section{Future Work}

TLB evictions can also be useful for predicting cacheline deadness. If a cacheline doesn't have any translations present in the TLB, is it likely dead? Not if you're just invalidating to change permissions, but otherwise probably yes. Write-updates clarify that and potentially make TLB-presence a more-useful deadness prediction factor. KPTI is a potential hazard here. What's the through-line on old TLB research \cite{clark_performance_1985} wondering if we should keep system pages vs. KPTI, I wonder?

Apply these learnings to a hardware system built with Minjie~\cite{xu_towards_2022} or Chipyard~\cite{amid_chipyard_2020}.

Page access tracking (or especially full coherence) could benefit security by allowing the TLB to unmap more aggressively and protect sensitive data from being accessed speculatively https://lwn.net/Articles/974390/

\endinput
%\include{conclusions}

%    3. Notes
%    4. Footnotes

%    5. Bibliography
\begin{singlespace}
\raggedright
\bibliographystyle{abbrvnat}
\bibliography{biblio}
\end{singlespace}

\appendix
%    6. Appendices (including copies of all required UBC Research
%       Ethics Board's Certificates of Approval)
%\include{reb-coa}	% pdfpages is useful here
\include{appendix}

\backmatter
%    7. Index
% See the makeindex package: the following page provides a quick overview
% <http://www.image.ufl.edu/help/latex/latex_indexes.shtml>


\end{document}
